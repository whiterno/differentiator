\documentclass{article}
\usepackage{ucs}
\usepackage[utf8x]{inputenc}
\usepackage[russian]{babel}
\usepackage{amsmath}
\usepackage[left=3cm,right=1.5cm,top=2cm,bottom=2cm]{geometry}
\usepackage{indentfirst}
\usepackage{amsfonts}
\usepackage{graphicx}

\begin{document}

    \section*{\Huge Number Theory}

        \subsection*{\huge Task 9.1}
        {\Large
            $1495 = 5\cdot13\cdot23$
            \par $3156^{792} \equiv (3156^{4})^{198} \equiv 1$ (mod 5)
            \par $3156^{792} \equiv (3156^{12})^{66} \equiv 1$ (mod 13)
            \par $3156^{792} \equiv (3156^{22})^{36} \equiv 1$ (mod 23) \par
            Эти выводы сделаны по малой теореме Ферма для простых чисел 5, 13, 23. \par
            По китайской теореме об остатках (далее КТО): \par
            $3156^{792} \equiv 1$ (mod 5 * 13 * 23 = 1495)
        }

        \subsection*{\huge Task 9.2}
        {\Large
            a)$x^2 \equiv x^2$ (mod p)\par
            $x^2 \equiv x^2 - 2px + p^2$ (mod p)\par
            $x^2 \equiv (p - x)^2$ (mod p)\par
            Таким образом количество квадратичных вычетов <= $\frac{p-1}{2}$\par
            Возьмем числа x $\neq$ y из \{1, ..., $\frac{p-1}{2}$\}, докажем, что невозможно $x^2 \equiv y^2$(mod p)\par
            $(x - y)(x + y) \equiv 0$ (mod p)\par
            |x - y| < p, x + y < p => (x - y)(x + y) не кратно p => среди чисел 1 ... p - 1 находится $\frac{p-1}{2}$ квадратичных вычетов\par
            б) Для 17: $\frac{17 - 1}{ 2} = 8$ квадратичных вычетов
        }

        \subsection*{\huge Task 9.3}
        {\Large
            Рассмотрим числа $2^n - 1, 2^n, 2^n+1$. Среди 3 последовательных чисел долнжо быть хотя бы 1 кратное 3, но так как $2^n - 1$ и $2^n + 1$ простые, то $2^n$ кратно 3, что невозможно
        }

        \subsection*{\huge Task 9.4}
        {\Large
            Пусть n > m:\par
            $(2^n - 1, 2^m - 1) = (2^n - 2^m + 2^m - 1, 2^m - 1) = (2^n - 2^m, 2^m - 1)$\par
            $(2^m(2^{n-m} - 1), 2^m - 1) = ((2^m - 1)(2^{n-m} - 1) + 1\cdot(2^{n-m}-1), 2^m - 1)$\par
            $(2^{n-m} - 1, 2^m - 1)$\par
            Повторяя аналогичные действия до конца (они ведь закончатся?...(Да, закончатся)) понимаем, что выполняется алгоритм Евклида для степеней =>\par
            $(2^n - 1, 2^m - 1) = 2^{(n,m)} - 1$
        }

        \subsection*{\huge Task 9.5}
        {\Large
            $S = n^2 + (n+1)^2 + (n + 2)^2 + (n+3)^2 + (n+4)^2$\par
            $S = n^2 + n^2 + 2n + 1 + n^2 + 4n + 4 + n^2 + 6n + 9 + n^2 + 8n + 16$\par
            $S = 5n^2 + 20n + 30 = 5(n^2 + 4n + 6)$\par
            То есть, если S квадрат некоего целого числа и кратно 5, то S кратно и 25. Таким образом $n^2 + 4n + 6$ кратно 5.\par
            Проверим, возможно ли такое. Так как остатки образуют цикл, проверим для $n \equiv 0, 1, 2, 3, 4$ (mod 5):\par
            $n \equiv 0$ (mod 5) => $n^2 + 4n + 6 \equiv 0 + 0 + 6 \equiv 1$ (mod 5)\par
            $n \equiv 1$ (mod 5) => $n^2 + 4n + 6 \equiv 1 + 4 + 6 \equiv 1$ (mod 5)\par
            $n \equiv 2$ (mod 5) => $n^2 + 4n + 6 \equiv 4 + 8 + 6 \equiv 3$ (mod 5)\par
            $n \equiv 3$ (mod 5) => $n^2 + 4n + 6 \equiv 9 + 12 + 6 \equiv 2$ (mod 5)\par
            $n \equiv 4$ (mod 5) => $n^2 + 4n + 6 \equiv 16 + 16 + 6 \equiv 3$ (mod 5)\par
            Таким образом S не может быть кратно 25, то есть S не квадрат
        }

        \subsection*{\huge Task 9.8 (ура, ура)}
        {\Large
            По малой теореме Ферма $1^{p-1}, 2^{p-1}...(p-1)^{p-1}$ сравнимы с 1 по mod p. Таким образом многочлен $x^{p-1} - 1$ имеет ровно p - 1 решение в $Z_p$. Значит имеет место тождество:
            $x^{p-1} -1 \equiv (x-1)(x-2)...(x - (p-1))$ (mod p)
            В частности:\par
            $-1 \equiv (-1)(-2)...(-p+1)$ (mod p)\par
            Если p = 2: (2 - 1)!  + 1 = 2 $\equiv$ 0 (mod 2)\par
            В иных случаях p нечетное, тогда:
            $-1 \equiv 1\cdot2 ... (p-1) = (p-1)!$ (mod p)\par
            $(p-1)! + 1 \equiv 0$ (mod p)
        }

\end{document}
